Afin de conclure cette étude, nous allons revenir sur les principaux défis auxquels nous
avons du faire face. Premièrement, bien que nous ayions obtenu des résultats très
satisfaisant concernant la modélisation des prix des caps, nos prédictions de volatilité
caps sont encore porteuse d'erreur d'un degré conséquent, surtout pour les volatilités à
court terme. En tentant d'améliorer la modélisation de ces volatilités de courte maturité,
c'est alors la qualité des prédictions de volatilité à plus long terme qui se
dégradait. Il fallut donc trouver un compromis entre la qualité de prédiction du long
terme et du court terme.  Deuxièmement, lorsque nous avons simulé nos trajectoires de taux
court, nous avons dû faire face à un taux court qui allait chercher des valeurs trop
élevées. Pour réduire ces trajectoires indésirables, nous avons du faire une savante
combinaison de poids et de bornes sur les paramètres afin d'obtenir des simulations dont
la volatilité n'était ni trop faible ni trop élevée et qui étaient bornées rapidement par
zéro. Troisièmement, certaines trajectoires de simulation du taux court furent
temporairement négatives et cet effet indésirable n'a pas pu être éradiqué.  Finalement,
comme modifier complètement la courbe des taux pour déterminer la durée ne pouvait être
envisagé, nous avons eu recours à une méthode de notre cru qui est inspirée de la durée
calculée sur un arbre binomial. Cependant, cette méthode n'est pas à toute épreuve. Comme
nous l'avons souligné, on obtient des valeurs assez faibles pour la durée du PO. Ceci
s'explique possiblement par le fait que le ``choc'' qu'on fait subir au taux d'intérêt ne
peut qu'être positif, puisqu'à une échéance d'un mois les taux ne peuvent que monter s'ils
veulent suivre la courbe forward. Encore une fois, au vu de l'absence d'une méthode
générale pour déterminer la durée d'un instrument dérivé d'un processus à taux court, nous
croyons avoir implémenté ce qui nous paraissait le plus raisonnable.

%%% Local Variables:
%%% mode: latex
%%% TeX-master: "rapport"
%%% End:
