Pour déterminer la durée, comme il ne nous a pas semblé exister de consensus sur la façon
de la calculer, nous avons procédé de la façon suivante (voir
\verb+duration2(N):mbs.py+). Tout d'abord nous avons déterminé les valeurs initiales des
IO et PO. Puis, afin de simuler un choc, nous nous sommes inspirés de la méthode employée
lors d'une tarification avec arbre binomial (voir \cite{veronesi}, p.~448). Parmis toutes
les trajectoires de taux court obtenues, nous n'avons retenues que celles qui tombaient à
une valeur inférieur à un écart type de la moyenne et celles supérieures à la même
quantité. À partir de ces taux initialement élevés ou bas, nous avons calculé la valeur
des IO et des PO un mois plus tard, que nous avons comparés à la valeur initiale. Ceci
nous a permis de simuler un choc sur la valeur du PO, observé au mois suivant, donc avec
un effet temporel relativement limité.

En somme, l'équation suivante a été employée:
\[
  D = -\frac{1}{V(0)}\frac{V(1/12)_{up} - V(1/12)_{down}}{\bar r_{up} - \bar r_{down}}
\]
pour les deux produits dérivés. Notons que $\bar r_{up}$ et $\bar r_{down}$ correspondent
à la moyenne des taux supérieurs et inférieurs.

On obtient avec une telle méthode, avec une simulation de $N=\num{100000}$ paramètres une
durée de \num{-1.0302} pour le IO et de \num{0.0316} pour le PO. Notons que ces résultats
sont cohérents pour les raisons évoquées plus haut.

Notons en passant que nous avons également été tentés de faire bouger la courbe zéro
coupon au grand complet, mais qu'un telle méthode est très lente à faire converger, et ce
pour plusieurs raisons. D'abord, la kurtose importante de la distribution de $r_t$
implique qu'une telle simulation peut mettre du temps à converger à une valeur suffisament
précise. Ensuite, le mouvement qu'on peut donner à la courbe est très faible, puisque
celle-ci commence déjà avec des valeurs faibles. De sorte qu'un mouvement si faible peut
être difficilement mesurable, notamment pour la raison évoquée ci haut. 


% Pour calculer la durée des IO et PO, la formule de la \textit{durée effective} pour MBS
% tirée de Veronesi p.301 est utilisée.

% \[
% D\approx-\frac{1}{P}\frac{P(+x bps)-P(-x bps)}{2*x bps}
% \]

% où $P$ représente la valeur du titre et $P(+x bps)$ et $P(-x bps)$ décrivent le prix du
% titre si la courbe de taux est décalée vers le haut ou vers le bas d'une quantité de $x$
% points de base. Nous obtenons les valeurs suivantes pour les durées de chacun des deux
% dérivés:


% \[
% D^{IO}\approx-\frac{1}{P^{IO}}\frac{P^{IO}(+1 bps)-P^{IO}(-1 bps)}{2*1 bps} \\
% D^{PO}\approx-\frac{1}{P^{PO}}\frac{P^{PO}(+1 bps)-P^{PO}(-1 bps)}{2*1 bps}
% \]



% Resultats

% La durée obtenue pour la IO est négative alors que la PO est positive. Ces résultats sont cohérents avec l'intuition décrite ci-dessus. D'ailleurs, en additionnant les deux durées de $D^{IO}$ et $D^{PO}$, nous retrouvons la durée de la MBS.

% \[
% D^{IO}+D^{IO}=X=D^{MBS}\approx-\frac{1}{P^{MBS}}\frac{P^{MBS}(+1 bps)-P^{MBS}(-1 bps)}{2*1 bps}
% \]


% %%% Local Variables:
% %%% mode: latex
% %%% TeX-master: "rapport"
% %%% End:
